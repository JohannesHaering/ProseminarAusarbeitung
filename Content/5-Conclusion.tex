\chapter{\iflanguage{ngerman}{Zusammenfassung}{Conclusion}}
\label{sec:results}
In conclusion, miniature robot systems are widely adopted in micro surgery. The research of  robot systems for surgeries began back in the 80s. Since then researches were able to improve their accuracy and scale them down. \\
Nowadays, commercially miniature robot systems for surgeries on the eye, brain, liver and many other organs  are available. Most of them are teleoperated over a master console but researcher working on automating the systems supported by the progress in other fields, like imaging and image processing technology. \\
With the advancing of miniature medical robot systems, the surgeons are able to heal diseases which were seen impossible to heal not long ago, like retinal vein occlusion. The robots are not only able heal more diseases but also the risk for the patient will be reduced massively because the probability of damaging tissue due to trembling or slipping is reduced through algorithms applied to the signal from the master console. \\
Current research mainly work on two problems. First, they try to improve the tools with the goal to wireless actuate the tool and to reach deeper and more narrow regions of organs with less injuries. Second, they try to help the surgeon with assisting algorithms. \\
Miniature robot system are more and more often used for micro surgeries due the many advantages. However, it is a new field of research and there are some issues left till the systems can be used for all micro surgeries. 