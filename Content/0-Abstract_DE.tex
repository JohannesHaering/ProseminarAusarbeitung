\Abstract{
Manuelle Mikrooperationen sind stark von der Genauigkeit des Chirurgen abhängig. Dieses Problem kann mit Miniatur-Robotersystemen gelöst werden. Mikroroboter werden in der Neurochirurgie, Augenheilkunde und Schwangerschaftshilfe eingesetzt. Für das Navigieren in engen Räumen im Gehirn wird ein halbflexibler Werkzeugschaft verwendet. Der Werkzeugschaft besteht aus Federn und steifen Scheiben aus Formgedächtnislegierungen. Die Flexibilität der Federn kann durch Wärme kontrolliert werden. \\
Mögliche Werkzeugspitzen zur Manipulation von Gewebe sind Mikroscheren und Milli-Greifer. Die Scherenblätter bestehen aus dünnen Titanblechen, die mit einer Feder verbunden sind. Der Milli-Greifer ist aus einer Formgedächtnislegierung hergestellt. Beide werden mit einem externen Magnetfeld betätigt. Bei Netzhautchirurgie werden aufgrund der geringeren Verschiebung des Fasernetzwerkes völlig flexible Werkzeuge eingesetzt. Im Bereich der Schwangerschaftshilfe wird ein Mikroroboter zum Navigieren und Halten eines Embryos in einem zur Implantation geeigneten Bereich eingesetzt, um die Schwangerschaftswahrscheinlichkeit zu erhöhen. \\
Die Miniaturroboter sind ferngesteuert. Die gemessene Bewegung des Chirurgen wird mit einem Computer verarbeitet, der dann den Roboter ansteuert. Dadurch kann das Zittern des Chirugen beseitigt und die Genauigkeit erhöht werden. Ein Problem mit Telesteuerung ist das fehlende haptische Feedback. Erste Automatisierungsansätze sind automatische Kollisionsvermeidung oder Echtzeit-Werkzeugerkennung. Auch werden erste Algorithmen zur Vorplanung der Operation entwickelt.
}
